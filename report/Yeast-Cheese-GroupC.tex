\documentclass[]{article}

%opening
\title{Group C: Report DNA/RNA Sequencing Course\newline Bread and Cheese}
\author{Max Suter, David Wittwer, (Audrey Minden)}

\begin{document}

\maketitle

\begin{abstract}
This is the final report of our work during the 2015 \emph{DNA/RNA Sequencing Course}, which is part of the joint master's programme in Bioinformatics of the Universities of Bern and Fribourg.

We present the result of two tasks: (1) finding novel regulators/targets of TORC1 in genomes of yeast mutants, and  (2) building and annotating a de-novo assembly of the genomes of five strains of Lactobacillus Paracasei. For both tasks, we're using free-of-charge and open source software tools which are well known in their respective field. (Quality control, assembling, SNP calling, annotation, visualization).

Using results from the first task, we are able to identify a candidate mutant, which shows interesting SNP mutations that are not yet known to be interferring with the TORC1 pathway.


Our main result from the second task is the newly assembled and annotated genomes of five strains of Paracasei.


\end{abstract}

\pagebreak
\tableofcontents

\pagebreak
\part{Identification of novel regulators and/or targets of TORC1 in yeast mutants}

\section{Introduction}
What is the goal?

Identify mutations that confer a growth phenotype after rapamycine treatment.

Asses, whether the involved proteins are already known to be involved in the TORC1 pathway.

If there are new candidates, we were trying to characterize their function as new regulators or targets of the TORC1 pathway.

\section{Data and Methods}
Genomic DNA libraries were prepared in advance and handed over to us in gzipped FASTQ file format.

\section{Results}

We identified the following SNPs to be interesting. Table xy shows these observed mutations.

\begin{center}
  \begin{tabular}{ | l | c | r | }
    \hline
    1 & 2 & 3 \\ \hline
    4 & 5 & 6 \\ \hline \hline
    7 & 8 & 9 \\
    \hline
  \end{tabular}
\end{center}

\subsection{Mutant I}
Already known to be interferring w/ the TORC1 pathway. (Refernece)

\subsection{Mutant II}
Already known to be interferring w/ the TORC1 pathway. (Refernece)

\subsection{Mutant III}
Shows some significant differences. ...interesting.

\section{Analysis and Discussion}

\pagebreak
\part{Paracasei}
\setcounter{section}{0}

\section{Introduction}
(Very) Short introduction to cheese making. Collaboration w/ Agroscope. Sequencing of different starters that they have in their library.

What is the goal of our work?

Which genes are associated with the growth phenotype?

What is the biochemical relation between VSC and the growth phenotype?


\section{Data and Methods}
We had the data from ...?

We used the following Pipeline...

Although the overall quality of the data was acceptable, we used sickle to correct and filter...

\section{Results}




\section{Analysis and Discussion}

\pagebreak
\appendix
\section{Appendix Some picture of data processing pipeline}
\section{Appendix Second picture of data processing pipeline}

\end{document}
